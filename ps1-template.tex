\documentclass[11pt]{article}

\input{preamble.tex}
\def\cO{\mathcal{O}}
\def\cA{\mathcal{A}}
\def\cS{\mathcal{S}}
\def\cD{\mathcal{D}}

\def\KeyGen{\mathsf{KeyGen}}
\def\Enc{\mathsf{Enc}}
\def\Dec{\mathsf{Dec}}
\def\pk{\mathsf{pk}}
\def\sk{\mathsf{sk}}
\def\ct{\mathsf{ct}}
\def\SKEnc{\mathsf{SK}\Enc}

\def\SZK{\mathbf{SZK}}
\def\Ptime{\mathbf{P}}

\def\classP{\mathcal{P}}
\def\classNP{\mathcal{NP}}
\def\classRP{\mathcal{RP}}
\def\classBPP{\mathcal{BPP}}
\def\classcoRP{co\mathcal{RP}}

\def\IO{\mathcal{O}}

\def\FHE{\mathsf{FHE}}
\def\ABE{\mathsf{ABE}}
\def\PPAD{\mathbf{PPAD}}


\newcommand{\PRFGen}{\mathsf{PRFGen}}
\newcommand{\PRFPunc}{\mathsf{Punc}}
\newcommand{\seed}{K}

\def\Adv{\mathcal{A}}


\def\PK{\mathsf{PK}}
\def\SK{\mathsf{SK}}
\def\MPK{\mathsf{MPK}}
\def\MSK{\mathsf{MSK}}

\def\secpar{\lambda}
\def\bitset{\{0,1\}}
\def\CRS{\mathsf{CRS}}




\def\display=1

\def\det{\mathsf{det}}
\newcommand{\IP}[2]{\langle {#1}, {#2} \rangle}

\def\Supp{\mathsf{Supp}}
\def\LWE{\mathsf{LWE}}
\def\ssLWE{\mathsf{ssLWE}}
\def\SIS{\mathsf{SIS}}

\def\vecu{\mathbf{u}}
\def\matV{\mathbf{V}}
\def\vecw{\mathbf{w}}
\def\matQ{\mathbf{Q}}
\def\matE{\mathbf{E}}
\def\matT{\mathbf{T}}
\def\matR{\mathbf{R}}

\def\R{\mathbb{R}}
\def\T{\mathbb{T}}
\def\Z{\mathbb{Z}}
\def\C{\mathbb{C}}

\def\matD{\mathbf{D}}
\def\matM{\mathbf{M}}

\def\lat{\mathcal{L}}
\def\basis{\mathbf{B}}
\def\ppd{\mathcal{P}}
\def\dist{\mathsf{dist}}

\def\eps{\epsilon}
\def\Adv{\mathcal{A}}
\def\maj{\mathsf{maj}}
\def\from{\gets}

\newcommand{\bra}[1]{\left| {#1} \right>}


\def\PRF{\mathsf{PRF}}
\newcommand{\LWR}[2]{\lfloor {#2} \rceil_{#1}}


\def\univ{\mathcal{U}}

\def\keygen{\mathsf{KeyGen}}
\def\enc{\mathsf{Enc}}
\def\setup{\mathsf{Setup}}
\def\dec{\mathsf{Dec}}
\def\eval{\mathsf{Eval}}

\def\trapsamp{\mathsf{TrapSamp}}
\def\dgsamp{\mathsf{DGSamp}}

\def\msk{\mathsf{msk}}
\def\mpk{\mathsf{mpk}}
\def\ek{\mathsf{ek}}

\def\sign{\mathsf{Sign}}
\def\verify{\mathsf{Verify}}

\def\hash{\mathsf{Hash}}

\def\IBE{\mathsf{IBE}}

%%%%%   NSDlathead

\renewcommand{\polylog}{\mathrm{polylog}}
\newcommand{\Rot}{\mathrm{Rot}}

\newcommand{\N}{\ensuremath{\mathbb{N}}}
\newcommand{\F}{\ensuremath{\mathbb{F}}}

\newcommand{\M}{\mathcal{M}}
\newcommand{\NN}{\mathcal{N}}
\newcommand{\V}{\mathcal{V}}

\renewcommand{\epsilon}{\varepsilon}
\newcommand{\grad}{\nabla}
\newcommand{\rank}{\mathrm{rank}}
\newcommand{\vol}{\mathrm{vol}}
\newcommand{\Var}{\mathrm{Var}}
\DeclareMathOperator*{\argmax}{arg\,max}
\DeclareMathOperator*{\argmin}{arg\,min}
\DeclareMathOperator*{\expect}{\mathbb{E}}

\renewcommand{\vec}[1]{\ensuremath{\boldsymbol{#1}}}
\newcommand{\gs}[1]{\ensuremath{\widetilde{\vec{#1}}}}
\newcommand{\problem}[1]{\ensuremath{\mathrm{#1}} }
\newcommand{\Tr}{\mathrm{Tr}}

% Problems
\newcommand{\gapsvp}{\problem{GapSVP}}
\newcommand{\usvp}{\problem{uSVP}}
\newcommand{\sivp}{\problem{SIVP}}
\newcommand{\gapsivp}{\problem{GapSIVP}}
\newcommand{\cvp}{\problem{CVP}}
\newcommand{\gapcvp}{\problem{GapCVP}}
\newcommand{\bdd}{\problem{BDD}}
\newcommand{\gdd}{\problem{GDD}}
\newcommand{\dgs}{\problem{DGS}}

\newcommand{\inner}[1]{\langle {#1} \rangle}

\def\XIO{xi\cO}
\def\size{\mathsf{size}}

\def\TK{\mathsf{TK}}
\def\TokO{\mathsf{Tok}\mathcal{O}}


    
\renewcommand{\O}{\mathcal{O}}
\newcommand{\zo}{\{0,1\}}

\newcommand{\inorm}[1]{\norm{#1}_\infty}

\def\LWE{\mathsf{LWE}}
\def\RLWE{\mathsf{Ring\text{-}LWE}}
\def\vecbt{\tilde{\vecb}}



\def\Prep{\mathsf{Prep}}
\def\Resp{\mathsf{Resp}}
\def\Query{\mathsf{Query}}
\def\Resp{\mathsf{Resp}}
\def\Process{\mathsf{Process}}
\def\Eval{\mathsf{Eval}}

\title{\textbf{6.5630 Advanced Topics in Cryptography} \\ Problem Set 1 \\ Due: October 7, 2024}
\author{}
\date{}

\begin{document}

\maketitle


\section{Minkowski is Tight [15 points]}
Minkowski's theorem is tight for general lattices. In particular, there is a family of lattices $\{\lat_n\}_{n\in\mathbb{N}}$ where $\lat_n$ lives in $n$ dimensions, and
\[ \lambda_1(\lat_n) \geq c\cdot \sqrt{n} \cdot \det(\lat_n)^{1/n} \]
where $c$ is a universal constant independent of $n$.
Show that such a family of lattices exists (your proof doesn't have to construct this family, you merely have to show existence). \textcolor{gray}{{\em Hint:} Try the SIS lattice. That is, pick a random $\matA \in \Zq^{n\times m}$ and look at the lattice $\Lambda^{\perp}(\matA) := \{\vecx: \matA \vecx = 0 \pmod{q}\}$. You can use the following fact:  if $\matA$ has rank $n$ over $\Zq$, then the determinant of $\Lambda^{\perp}(\matA)$ is $q^{n}$.}

\medskip\noindent
\textbf{Optional.} Same problem except show an explicit construction of such a family of lattices $\{\lat_n\}_{n\in\mathbb{N}}$.


\section{(Our Analysis of) LLL is Tight [15 points]}
(For a refresher on the LLL algorithm, look at the notes for Lecture 2.)

\medskip\noindent
Let $\delta = 3/4$. Find a $\delta$-LLL reduced basis $\vecb_1,\ldots,\vecb_n \in \mathbb{R}^n$  such that $\vecb_1$ is longer than the shortest vector by a factor of $\Theta(2^{n/2})$. In other words, our analysis of the LLL algorithm using LLL reduced bases is tight.



\section{Cheap Gaussian Sampling? [15 points]}

Consider the following algorithm for sampling from the zero-centered discrete Gaussian distribution $D_{\lattice,s}$. Assume we have a good basis $\matB$ of $\lattice$. The algorithm samples a point from the continuous Gaussian distribution $\rho_{s}(x)/s^n$, rounds it to a nearby lattice point (say, using Babai’s rounding algorithm), and outputs
the result. 

Show that the output of this algorithm is statistically quite far from $D_{\lattice,s}$, even for
radii $s$ that are polynomially bigger than the length of the given basis.
\textcolor{gray}{{\em Hint:} Try $\mathbb{Z}$.}



\section{Spooky Encryption (20 points)}
\newcommand{\Spook}{\mathsf{Spook}}

Fully homomorphic encryption tells us how to transform an encrypted input $c \in \Enc_{\pk}(x)$ into an encrypted output $c' \in \Enc_{\pk}(f(x))$ for any polynomial-time computable function $f$. Suppose you are now given $$c_1 \in \Enc_{\pk_1}(x_1) \mbox{  and  } c_2 \in \Enc_{\pk_2}(x_2)$$ for {\em independent} public keys $\pk_1$ and $\pk_2$ and some $x_1,x_2$. 


\begin{itemize} 
\item  Could one now compute an encryption of $f(x_1,x_2)$ under either $\pk_1$ or $\pk_2$?  Show a function $f$ such that being able to do so for any $x_1,x_2$ will violate the IND-CPA security of the encryption scheme. 

\item Starting with the GSW FHE scheme we saw in class, construct an FHE scheme where one can produce two ciphertexts $c_1'$ and $c_2'$ such that 
$$ \Dec_{\sk_1}(c_1') \oplus \Dec_{\sk_2}(c_2') = f(x_1,x_2)$$
\end{itemize}


\section{If Pigs Fly...? [20 points]}
The goal of private information retrieval (PIR) is for a client to obtain the $i$'th bit of a database $D \in \zo^N$ from server, without the server learning anything about $i$. 
\begin{definition}[Private Information Retrieval]
    A PIR scheme consists of four (potentially randomized) algorithms $\Prep$, $\Query$, $\Resp$, and $\Dec$, which are run in the following order:
\begin{enumerate}
    \item \textbf{Prep.} The server preprocesses the dataset by computing $\tilde{D} \gets \Prep(D)$. (This step is done once and for all by the server.)
    \item \textbf{Query.} To query index $i \in [N]$, the client computes $(q, s) \gets \Query(i, 1^\lambda)$ and sends $q$ to the server. ($s$ is some private state that the client keeps to herself.)
    \item \textbf{Respond.} The server sends $a \gets \Resp(q, \tilde{D})$ back to the client. (Here, $\Resp$ has random access to its inputs, so it can potentially run in sublinear time)
    \item \textbf{Decode.} Client computes $b  \gets \Dec(a, s)$.
\end{enumerate}
We require that the scheme has the following two properties: \begin{itemize}
    \item  \textbf{Correct:} In the notation above, for all $i$, we have $b = D_i$ with probability one. 
    \item \textbf{Secure:} for all $i, i' \in [N]$, the distributions $\Query(i, 1^\lambda)$ and $\Query(i', 1^\lambda)$ are computationally indistinguishable. 
\end{itemize}
\end{definition}

\noindent
We ask you to prove the following:
\begin{enumerate}
    \item Show that if no preprocessing is done (i.e., $\tilde{D} = D$) by a PIR scheme, then $\Resp(q, \tilde{D})$ needs to run in $\Omega(N)$ time. 
    \item Assuming a ``strongly preprocessable'' (defined below) homomorphic encryption scheme exists, construct a PIR scheme where $\Query$, $\Resp$, and $\Dec$ run in $\poly (\log N)$ time and $\Prep$ runs in $\poly(N)$ time.
\end{enumerate}

\begin{definition}[Strongly Preprocessable Homomorphic Encryption Scheme]
    A fully homomorphic encryption scheme is strongly preprocessable if there are deterministic algorithms $\Process$ and $\Eval$ such that both of the following hold: 
\begin{itemize}
    \item given a circuit $C: \zo^n \to \zo$ of size $s$ (which can be much larger than $n$), $\Process(C)$ runs in $\poly(s,n)$-time and outputs a string $\tilde{C}$, and 
    \item if $ct$ is an encryption of $x \in \zo^n$, then $\Eval(ct, \tilde{C})$ runs in $\poly(n)$-time (note this is independent of s!!)  and outputs an encryption of $C(x)$. Here, $Eval$ is given random access to $\tilde{C}$.
\end{itemize}
\end{definition}




\end{document}

